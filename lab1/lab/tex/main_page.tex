\section{Оглавление}
(Ссылки кликабельны)\\
\hyperlink{p1}{Задание.....................................................................................................................................................3}\\
\hyperlink{p2}{Выполнение...............................................................................................................................................3}\\
\hyperlink{p3}{Пример исполнения процедуры...............................................................................................................4}\\
\hyperlink{p4}{Вывод........................................................................................................................................................4}\\
\newpage
\section{Задание}

\hypertarget{p1}{Используя}сведения из системных каталогов, получить информацию об используемой базе данных. Полученную информацию представить в следующем формате:
\begin{verbatim}
DBID: 			00000001
NAME:			orbis
CREATED:		2014-09-10 00:00:00
LOG_MODE:		NOARCHIVELOG
OPEN_MODE:		MOUNTED
PROTECTION_MODE:	MAXIMUM PROTECTION
\end{verbatim}

Так как программа реализована для PostgreSQL, а вариант подразумевает работу с OracleDB было принято решение заменить некоторые части задания (невозможные для реализации), на похожие. Те, аналоги которых небыли найдены - заменены в отношении 1к3 на возможные. \\
Таким образом в выводе появились поля:
\begin{verbatim}
ARCHIVE_MODE:                    off
LOGGING_COLLECTOR:               on
ENCODING:                        UTF8
MAX_CONNECTIONS:                 256
TIMEZONE:                        UTC
DATA_DIRECTORY:                  <NULL>
DEFAULT_TRANSACTION_ISOLATION:   read committed
WAL_LEVEL:                       replica
\end{verbatim}

\section{Выполнение}
\hypertarget{p2}{Код процедуры с последующим вызовом}
\lstinputlisting[language=SQL]{src/rshd.sql}
\section{Пример исполнения процедуры}
\hypertarget{p3}{Вызов функции из терминала}
\lstinputlisting[language=SQL]{src/res.sql}
\section{Вывод}
\hypertarget{p4}{Выполняя} первую лабараторную работу по дисциплине Распределенные системы хранения данных я изучил работу с системным каталогом и параметрами БД.

